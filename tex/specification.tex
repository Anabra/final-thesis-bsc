\documentclass[main.tex]{subfiles}
\begin{document}
	
	In this chapter we will outline the main aspects of the implemented software. First, a general specification of the extension elimination problem will be discussed, then we will review the currently existing methods and tools, and finally, we will specify the exact requirements the program should comply to.
	
	\section{Problem specification}
	
	In order to eliminate all unnecessary extensions from a module, first we must define what \emph{unnecessary} exactly means. First and foremost, after removing an \emph{unnecessary} extension from our module, we want our program to behave exactly as it did before. To be more specific, we want the refactored program to be semantically equivalent to the original one. A relatively convenient way to check this, is to determine whether every single language element in the module has the exact same type as before the refactoring. So, given a module compiles with and without a certain language extension, we deem the extension \emph{unnecessary}, if and only if removing it has no effect on the calculated types of the language elements present in module.
	
	It is also important to define when does a given language element require a certain extension. We can only give a relatively vague definition in the general case: we say that a language element requires a certain extension, if only if it uses at least one feature enabled by the extension. Also, we want to determine the minimal language element requiring the extension. So, in the example below, we want to only find the highlighted code segment for \ext{ViewPatterns}, and not say, the entire function definition.
	
	\vspace{-0.3cm}
	\begin{haskell}
		repeatHead  %\colorbox{lightblue}{(safeHead -> Just x)}% = repeat x
		repeatHead = []
	\end{haskell}
	
	Using the above two definitions, we can define the exact specifications of our software. The implemented program should be able to remove all unnecessary extensions from a Haskell module, as well as highlight all language elements requiring at least one extension.
	
	\vspace{-0.3cm}
	
	\section{State of the art}
	
	Some compilers of other languages like \pilcode{C} or \pilcode{C++} also support certain language extensions. For instance, the GNU C compiler~\cite{gnu-docs} offers several new features for the language. It also has a compiler flag that can determine which extensions are used in the program. Since most of these language extensions require using macros, or predefined functions, checking whether they are actually used can be determined at the time of preprocessing. However, for a few of these extensions like nested functions, a simple syntactical analysis is required. Furthermore, these extensions cannot be enabled or disabled by the user, they come with the compiler by default.
	
	As for Haskell, there are only a few tools that can help programmers eliminate unused language extensions. GHC can only determine if a certain extension is needed, but is not enabled. In this case, the compilation fails, and the compiler shows an error message. Unfortunately, GHC only reports the first few missing extensions, and then it instantaneously aborts compilation. So collecting the extensions from the error messages is not an option.
	
	There is only a single tool which has support for removing unused extensions. This tool is called HLint~\cite{hlint-bib}. HLint is a linter for Haskell, which can suggest possible source code improvements. The tool operates in a purely syntactic way. It uses \pilcode{haskell-src-exts}~\cite{haskell-src-exts} to parse the Haskell source code, then it looks for patterns in the syntax tree to identify code smells, reduce code duplication or even find unnecessary language extensions. However, since the tool only has access to syntactic information, it cannot say anything about extensions requiring semantic analysis.
	
	\vspace{-0.3cm}
	
	\section{Requirements specification}
	
	The method should be implemented as a program transformation, possibly integrated into a graphical text editor. The user should be able to apply the refactoring to a single module or to a complete Haskell package as well. Applying the refactoring to a module or a project should eliminate all unnecessary extensions from it. A query option should also be implemented which would highlight the language elements in a given module using at least one extension. The latter feature could help programmers identify the reason why a certain extension is enabled, aiding them in the process of manually transforming the source code so that it no longer requires the given extension. The refactoring and the query should be able to handle both syntactic and semantic extensions as well.
	
\end{document}