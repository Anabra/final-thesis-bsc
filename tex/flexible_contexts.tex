\documentclass[main.tex]{subfiles}
\begin{document}
	
	In Haskell types can have contexts associated with them. These contexts constrain the possible values the type can be instantiated with.
	
	\begin{oneLineHaskell}
		elem :: Eq a => a -> [a] -> Bool
	\end{oneLineHaskell}
	
	In the above example the function \ilcode{elem} only works on types satisfying the \ilcode{Eq} constraint, i.e. they have an equality operation defined on them. In Haskell 98 we can only define such constraints that have only type variable arguments. So the following type signature is illegal:
	
	\begin{oneLineHaskell}
		listEq :: Eq [a] => [a] -> [a] -> Bool
	\end{oneLineHaskell}
	
	Enabling \ext{FlexibleContexts} lifts this restriction, and allows users to define more complex constraints. More precisely \ext{FlexibleContexts} have to be enabled, if there are any bindings in the module whose left hand-side has type in which there is a type class constraint, and at least one of its type arguments does not have a type variable head. 
	
	This also applies to any element of the abstract syntax tree that is explicitly type annotated. Note that the constraint must be a type class constraint. The rule above neither applies to equality constraints, nor to type family applications of result kind \ilcode{Constraint}. It is also worth emphasizing that there may not be any syntactic evidence of such complex constraints, and the extension might still have to be enabled. Program~code~\ref{code:flexible-contexts-inferred} illustrates this situation.
	
	\begin{codeFloat}
		\noindent
		\begin{minipage}{0.47\textwidth}
			\begin{haskell}
				{-# LANGUAGE FlexibleContexts #-}
				module A where
				
				class C a where
				
				f :: C [a] => a -> a
				f = id
			\end{haskell}
		\end{minipage}
		\hfill
		\begin{minipage}{0.50\textwidth}
			\begin{haskell}	
				{-# LANGUAGE FlexibleContexts #-}
				module B where	
					
				import A
								
				
				g x = f x
			\end{haskell}
		\end{minipage}
		\caption{Inferred type requiring \ext{FlexibleContexts}}
		\label{code:flexible-contexts-inferred}
	\end{codeFloat}

	As we can see, in \ilcode{module A}, \ilcode{f} has type which requires \ext{FlexibleContexts}. This function is imported into \ilcode{module B} and is used in such a way, that \ilcode{g}'s inferred type will also require the extension. This is a very simplified example, but in practice, many similar situations arise regularly.
	
	As explained earlier, all bindings whose left hand-side have type in which there is a type class constraint, and at least one of its type arguments does not have a type variable head will be considered as evidence for the necessity of \ext{FlexibleContexts}. However, such identifiers on the \textbf{right-hand} sides of bindings not necessarily require \ext{FlexilbeContexts}, so we can only give hints about them. More precisely, those identifiers whose type requires \ext{FlexibleContexts}, but stand on the right-hand side of a binding, will only be considered as hints for the necessity of \ext{FlexibleContexts}.
		
\end{document}