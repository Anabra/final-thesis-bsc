\documentclass[main.tex]{subfiles}
\begin{document}
	
	\ext{TypeFamilies} opens up the possibilities for type-level programming in Haskell by introducing type level functions, type-indexed data types and type equalities. This extension can also be considered as an alternative to \ext{FunctionalDependencies}. However, as opposed to the relational style of \ext{FunctionalDependencies}, \ext{TypeFamilies} allows for a more functional style of type-level programming.
	
	The detection of type family declarations, associated types in classes and explicit occurrences of type equalities in type signatures can be solved using the same technique as with syntactic checkers. We only need to find certain nodes in the syntax tree, and mark them as evidence for \ext{TypeFamilies}. However, in the special case of type equalities, we cannot determine whether it requires \ext{TypeFamilies} or \ext{GADTs}. The checker resolves this issue by associating such nodes with an extension formula. Precisely, all explicit type equality occurrences, and all bindings whose left hand-side have type which contains a type equality constraint will be marked as evidence for $\ext{TypeFamilies} \lor \ext{GADTs}$.
	
	As for type equality constraints present in the inferred types of language elements standing on the right-hand side of a binding, we will follow the same methodology as presented in Section~\ref{flexible-contexts}. We will mark these nodes as hints for $\ext{TypeFamilies} \lor \ext{GADTs}$.
	
	It is worth mentioning, that neither \ext{TypeFamilies} nor \ext{GADTs} can be completely eliminated using our current technique, because both of them imply \ext{MonoLocalBinds}. As discussed earlier in Section~\ref{mono-local-binds-complication}, \ext{MonoLocalBinds} cannot be safely and efficiently removed from any module.	
	
\end{document}